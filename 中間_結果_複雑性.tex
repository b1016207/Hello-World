

\documentclass{jsarticle}
\usepackage[dvipdfm]{graphicx,color}
	\begin{document}
 \section{複雑性を考慮した性能比較とその結果}

本項では、複雑性を考慮した性能比較について、また、その結果について説明する。



\subsection{複雑性の定義について}

より人に扱いやすい戦略を定義する為に、A.N. Kolmogorov氏の『On tables of random numbers』を参考にし、戦略の複雑性を次のように設定した。
まず、戦略の文字列を圧縮する。圧縮の方法は、「連続する文字+連続して文字が出た回数」を合わせたものとした。例として、「HHSSSHHHHH」という10字からなる文字列を圧縮すると、「H2S3H5」となり、圧縮した後の文字列は6字となる。この時、連続して文字が出た回数が2桁になったとしても、ここでは1字として数える。



\subsection{各戦略の複雑性}

この圧縮の方式を各戦略に行い、それぞれの圧縮された後の文字列の長さを元の長さで割ったものを複雑性とした。用意した戦略は次のような8行の配列とし、それぞれの行に圧縮を行った。

\includegraphics[width=15cm,bb=0 0 602 281]{1.png}

表◯.基本戦略の戦略表


今回用意した戦略を、全て圧縮したのが以下の表である。

\includegraphics[width=15cm,bb=0 0 602 261]{2.png}

表◯.各戦略の圧縮した後の文字列と複雑性


\subsection{各戦略の性能評価}

そして、今回はその複雑性を用いて、各戦略の性能比較を行った。
性能の基準は以下の二通りを用意した。
1. (勝率) ÷ (複雑性)
2. (勝率) - (複雑性)
この評価基準に従って、1デックの時と無限デックの時の性能を表にした。


\includegraphics[width=15cm,bb=0 0 602 279]{3.png}

表◯.1デックの時の各戦略の性能

\includegraphics[width=15cm,bb=0 0 602 283]{4.png}

表◯.無限デックの時の各戦略の性能

\subsection{各戦略の比較結果}

各戦略を比較し、次のような結果を得た。
勝率のみを考慮した場合、一定の数字以上でスタンドする戦略よりも、基本戦略とそれを改変した戦略の方が有意に高い勝率だった。
また、基本戦略と改変1、改変2のそれぞれの戦略間には有意な差が見られなかった。
複雑性を考慮して性能を評価した場合、基準値を15に設定した戦略が一番優秀であった。

\subsection{今後の課題}

今回は、複雑性の設定を手動で行い、検証する時間もあまり取らなかったので、評価基準が正確ではない可能性がある。
今後、この評価基準をどのように調整するかは検討の余地がある。


\end{document}